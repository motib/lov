\subsection{Clone}\label{inher.08}

\textbf{Concept} Assigning a variable containing a reference to another
of the same type merely copies the 
reference so that two fields refer to the same object. The method 
\texttt{clone} is used to copy the content of an object into a new one. 
\texttt{clone} is defined in class \texttt{Object} and can be
overridden in any class definition.

\prg{Inheritance08}
\prgl{inheritance}{Inheritance09}

\texttt{clone} is overridden in class \texttt{Particle}. 
The class must implement the interface \texttt{Cloneable}, the 
method of the superclass should be called, and we have to take into 
account that the method might raise an exception. The method returns
the object returned by superclass method after calling \texttt{newPosition}.
\begin{itemize}
\item An object of class \texttt{Particle} is allocated and its reference 
assigned to the field \texttt{p1}. 
\item An assignment statement copies this reference to the field \texttt{p2}. 
Check that they have the same value.
\item The method \texttt{newPosition} is called on \texttt{p1}, but the 
value of \texttt{p2.position} is also changed, showing that the two fields 
point to the same object.
\item An object of class \texttt{Particle} is obtained by calling \texttt{p1.clone()} and its reference assigned to the field \texttt{p3}. Since \texttt{clone} returns a value of type
\texttt{Object}, it must be cast to type \texttt{Particle} before the assignment.
Check that the objects referenced by \texttt{p1} and \texttt{p3} have different values.
\item Calling \texttt{p3.newPosition} changes only the field in the object
referenced by \texttt{p3} and not the separate object referenced by \texttt{p1}.
\end{itemize}

\textbf{Exercise} The method \texttt{clone} can perform arbitrary 
computation. Modify the program so that new objects are initialized with 
the absolute value of the field of the object that is being cloned.
