\subsection{Overloading vs. overriding}\label{inher.09}

\textbf{Concept} \emph{Overloading} is the use of the same method name with a 
\emph{different} parameter signature. \emph{Overriding} is the use in a subclass 
of the same method name with the \emph{same} parameter signature as a method 
of the superclass.

\prg{Inheritance09}
\prgl{inheritance}{Inheritance09}

The method \texttt{newPosition(int delta)} is declared in \texttt{Particle} and
\emph{overridden} in \texttt{AParticle}. It is also \emph{overloaded} by a method
with the same name takes a parameter of type \texttt{double}.
\begin{itemize}
\item After allocating three objects \texttt{p}, \texttt{a1} and \texttt{a2}, \texttt{newPosition} is called on  each one. 
\item \texttt{p.newPosition} calls the method declared in class 
\texttt{Particle}.
\item \texttt{a1.newPosition} calls the method declared 
in class \texttt{AParticle} that overrides the method in 
\texttt{Particle}. 
\item \texttt{a2.newPosition} calls the overloaded method 
because the actual parameter is of type \texttt{double}.
\end{itemize}

\textbf{Exercise} At the end of the program add an assignment \texttt{p = 
a1}. Add the method invocations \texttt{p.newPosition(10)} and 
\texttt{p.newPosition(10.0)} in the main method. Explain what happens.
