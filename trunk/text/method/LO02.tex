\subsection{A method returning a value}\label{method.02}

\textbf{Concept} When a method that is declared with a return type is called,
it allocates memory for its parameters and local variables, executes
its statements and then returns a value of the type. 
The call is a statement constructed
from the name of the method followed by a list of actual parameters;
the call is an expression and can appear wherever an expression is allowed.

\prg{Method02}
\prgl{method}{Method02}

\begin{itemize}
\item The variables \texttt{x} and \texttt{y} are allocated and initialized;
the variable \texttt{max} is allocated but not initialized.
\item An assignment statement is executed: the expression on the right hand side is
a method call including the values of the actual parameters \texttt{x} and \texttt{y}.
\item Memory is allocated for the formal parameters of the method and the local variables.
This is called an \emph{activation record} and is displayed by \jel{} in the upper left
hand part of the screen labeled \texttt{Method Area}. The new activation record hides
the previous ones which are no longer accessible.
\item The actual parameters are used to initialize the formal parameters in the activation
record.
\item The statements of the method are executed.
\item When the statement \texttt{return b} is executed, the value of \texttt{b} is used
for the value to be returned.
\item The method \emph{returns} and the activation record is deallocated.
\item The value returned becomes the value of the expression assigned to the
variable \texttt{max}.
\item The value of \texttt{max} is printed.
\end{itemize}

\textbf{Exercise} Write the body of the main method as one statement.
