\subsection{Returning locally instantiated objects}\label{method.09}

\textbf{Concept} When a method terminates, its activation record is deallocated.
However, if an object has been instantiated \emph{within the method},
a reference to the object can be returned to the calling method.

\prg{Method09}
\prgl{method}{Method09}

This program computes the cost of a song as the product of its length
in seconds and the price per second. A class \texttt{Song} is defined
to encapsulate the field \texttt{seconds} and the method \texttt{computePrice}.
The method \texttt{double} in the \texttt{main} method receives a reference
to an object of class \texttt{Song} as a parameter and returns a reference
to an new object of class \texttt{Song} whose field \texttt{seconds} is twice as large.

\begin{itemize}
\item Two objects of class Song are instantiated and references to them
are assigned to the variables \texttt{song1} and \texttt{song2}.
\item The method \texttt{doubleSong} is called with an actual parameter
that is a reference \texttt{song1} to an object of class \texttt{Song}.
The actual parameter is used to initialize the formal
parameter; check that \texttt{song1} and \texttt{s1} reference the same object.
\item \emph{Within the method}, the \texttt{seconds} field of the object referenced
by \texttt{s1} is used to instantiate a new object whose reference is assigned
to the variable \texttt{d} of class \texttt{Song}.
\item The method returns the reference to the object contained in \texttt{d};
although \texttt{d} disappears when the activation record is deallocated,
the object still exists as does the reference that is returned.
\item The returned reference is assigned to the variable \texttt{longSong};
check that \texttt{song1} and \texttt{longSong} reference \emph{different} objects!
\end{itemize}

\textbf{Exercise} Replace the last line of the program by:
\begin{quote}
\texttt{song1 = doubleSong(song1);}
\end{quote}
and explain precisely what happens.

