\subsection{Recursion}\label{method.04}

\textbf{Concept} \emph{Recursion} occurs when method calls itself.
There is nothing at all mysterious about recursion!
Each call simply creates a new activation record on the stack.
However, to ensure that the recursive calls terminate,
eventually, some call of the method should return without
invoking itself once again.

\prg{Method04}
\prgl{method}{Method04}

The standard example of a recursive method is one
that computes the factorial function:
\begin{displaymath}
n! = n\cdot{} (n-1) \cdot{} \cdots{} \cdot{} 2 \cdot{} 1 = n \cdot{} (n-1)!
\end{displaymath}
The recursion is terminated by defining $n!=1$ for $n\leq 1$.

\begin{itemize}
\item The \texttt{main} method calls the method \texttt{factorial}
with the actual parameter 5. This creates an activation record with
the formal parameter \texttt{n} initialized to 5.
\item To compute the expression in the second return statement,
the method \texttt{factorial} is called again, this time with the actual
parameter equal to $5-1=4$.
\item The sequence of recursive calls continues five times,
each one allocating a new activation record with a new variable \texttt{n}.
\item Finally, \texttt{factorial} is called with actual parameter 1.
This call creates a new activation record as usual, but does not
cause \texttt{factorial} to be invoked again. Instead,
the value 1 is returned and the activation record is deallocated.
Just before the method returns, select the tab \texttt{Call Tree} above the graphic display;
the sequence of calls from \texttt{main} to the sequence of recursive calls is displayed.
Select \texttt{Theater} to return to the animated display.
\item The recursive sequence \emph{unfolds}: each returned value
is used to compute a new value to be returned by that call of \texttt{factorial}.
\item Finally, the value 120 is returned to the \texttt{main} method and printed.
\end{itemize}

\textbf{Exercise} Write a recursive method to compute the n'th Fibonacci number: 
\begin{displaymath}
fib(0) = 0, fib(1) = 1, fib(n) = fib(n-1) + fib(n-2) \;\;\mathit{for}\;\; n > 1.	
\end{displaymath}


\textbf{Exercise} Write a more efficient nonrecursive method for the same function.
