\subsection{Calling methods on an object}\label{method.05}

\textbf{Concept} Nonstatic methods defined in a class must be invoked
\emph{on} an object of the class. A reference to the object becomes
an \emph{implicit} actual parameter that initializes a formal variable called
\texttt{this} in the method. The variable \texttt{this} need not be 
explicitly mentioned when accessing fields of the object unless there is an ambiguity.

\prg{Method05}
\prgl{method}{Method05}

This program computes the cost of a song as the product of its length
in seconds and the price per second. A class \texttt{Song} is defined
to encapsulate the field \texttt{seconds} and the method \texttt{computePrice}.

\begin{itemize}
\item Two objects of class Song are instantiated and references to them
are assigned to the variables \texttt{song1} and \texttt{song2}.
\item The method \texttt{computePrice} is called \emph{on} the object referenced by \texttt{song1}. In \jel{} this is visualized by an arrow to the object
placed in the \texttt{Expression Evaluation Area} followed by a period and
the method name and parameters.
\item An activation record is allocated containing two formal parameters:
\texttt{this} is initialized by the implicit reference and \texttt{pricePerSecond}
is initialized from the actual parameter.
\item The reference in the parameter \texttt{this} is used to obtain 
the value of the field \texttt{seconds}. An expression is evaluated and
its value returned.
\item The activation record is deallocated and the value returned is stored in the variable \texttt{price1}.
\item A second call to the method is executed in exactly the same way,
except that it is called \emph{on} the object referenced by \texttt{song2}.
\item The values of \texttt{price1} and \texttt{price2} are printed.
\end{itemize}

\textbf{Exercise} Modify the method so that the formal parameter is \emph{also} 
named \texttt{seconds}. Yes, it can be done! 
(Hint: read the \textbf{Concept} paragraph above.)
