\subsection{Objects as parameters}\label{method.07}

\textbf{Concept} A reference to an object can be an actual parameter whose
corresponding formal parameter is declared to be of the same class.
As with all parameters, the \emph{value} of actual parameter is used
to initialize the formal parameter, but since it is a reference that is
passed, the method that is called can access fields and methods of the
object. This is called \emph{reference semantics}.

\prg{Method07}
\prgl{method}{Method07}

This program computes the cost of a song as the product of its length
in seconds and the price per second. A class \texttt{Song} is defined
to encapsulate the field \texttt{seconds} and the method \texttt{computePrice}.
The method \texttt{getPrice} in the \texttt{main} method receives an object of class
\texttt{Song} as a parameter and calls \texttt{computePrice}.

\begin{itemize}
\item Two objects of class Song are instantiated and references to them
are assigned to the variables \texttt{song1} and \texttt{song2}.
\item The method \texttt{getPrice} is called with two parameters:
the first is a reference \texttt{song1} to an object of class \texttt{Song}, while the second
is a value of type double. The actual parameters are used to initialize the formal
parameters; check that \texttt{song1} and \texttt{s} reference the same object.
\item Since the formal parameter \texttt{s} receives a reference to an object
of class \texttt{Song} (in this case \texttt{song1}), it can be used to call
the method \texttt{computePrice} declared in the class.
\item The method returns a value that is assigned to \texttt{price1}.
\item A second call to the method is executed exactly the same way,
except that the actual parameter is the reference contained in \texttt{song2}.
\item The values of \texttt{price1} and \texttt{price2} are printed.
\end{itemize}

\textbf{Exercise} Modify the program so that discount does not use the explicit
parameter \texttt{s}.
