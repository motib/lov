\subsection{Array objects}\label{array.01}

\textbf{Concept} An array is created in three steps: first a variable of 
an array type is declared; then the array is allocated; finally, the 
elements of the array are given values. The syntax for accessing an array 
\texttt{a} is \texttt{a[i]}, and the field \texttt{a.length} gives the 
length of the array, so that if we modify the program by changing the size 
of the array the rest of the program need not change.

\prg{Array01A}
\prgl{array}{Array01A}

The program creates an array in the three steps described above.

\begin{itemize}
\item Initially, the variable \texttt{fib} of type integer array (denoted 
\texttt{int[]}) is allocated and contains the null value.
\item \texttt{new fib[7]} creates an array object with its seven fields having the default 
integer value zero; then the reference to the object is returned and 
stored in the variable \texttt{fib}.
\item The length field of the array is displayed above the cells for the elements.
\item A \texttt{for} loop is used to assign values to each element of the 
array.
\item The thin white lines show the constants and expressions that are used
as indices into the array.
\item \emph{Automatic dereferencing}: 
Although expressions like \texttt{fib[i-2]} seem to indicate that 
\texttt{fib} is being indexed, \texttt{fib} contains a reference to an array;
an implicit operation of dereferencing is carried out to obtain the
array itself from the reference and the index \texttt{[i-2]} is then
applied to that array.
\end{itemize}

\textbf{Concept} It is possible to combine the first two steps in creating an array: 
declaring the array field and allocating the array object.

\prg{Array01B}
\prgl{array}{Array01B}

This program combines the declaration of the
array field with its allocator. We have used the constant \texttt{SIZE} to 
specify the size of the array; this makes it easier to modify 
the program; nevertheless, \texttt{fib.length} is still used in the 
executable statements.
\begin{itemize}
  \item Initially, the static variable \texttt{SIZE} is created in the constant
  area and given its value.
  \item The execution of the program is as before.
\end{itemize}

\textbf{Exercise} Modify the program so that the fibonacci sequence appears in
reverse order.
