\section{Learning Objects for Arrays}\label{s.arrays}

\textbf{Concept} An array is a sequence of elements of the same type;
the type of the elements can be a primitive types such as \texttt{int},
or a predefined or user-defined class type. To access an element of an array, an index is given; this may be any expression of type 
\texttt{int}, including an integer literal or a variable.

\begin{center}
\begin{tabular}{|c|l|l|c|}
\hline
LO & Topic  & Java Files (.java) & Prerequisites \\\hline
\ref{array.01} &  Array objects                       & Array01A, B &\\\hline
\ref{array.02} &  Array initializers                  & Array02  &  1\\\hline
\ref{array.03} &  Passing arrays as parameters        & Array03  &  2\\\hline 
\ref{array.04} &  Returning an array from a method    & Array04  &  2\\\hline
\ref{array.05} &  Array assignment can create garbage & Array05  &  4\\\hline
\ref{array.06} &  Two-dimensional arrays              & Array06  &  3\\\hline
\ref{array.07} &  Arrays of arrays                    & Array07  &  6\\\hline
\ref{array.08} &  Ragged arrays                       & Array08  &  6\\\hline
\ref{array.09} &  Arrays of objects                   & Array09  &  3\\\hline
\end{tabular}
\end{center}

The example used in LO~\ref{array.01} through LO~\ref{array.01} is to fill an array 
with a sequence of fibonacci numbers (0,1,1,2,3,5,8). The programs for 
LO~\ref{array.05} through LO~\ref{array.08} concern matrices. 
The program for LO~\ref{array.09} is explained there.

\subsection{Array objects}\label{array.01}

\textbf{Concept} An array is created in three steps: first a variable of 
an array type is declared; then the array is allocated; finally, the 
elements of the array are given values. The syntax for accessing an array 
\texttt{a} is \texttt{a[i]}, and the field \texttt{a.length} gives the 
length of the array, so that if we modify the program by changing the size 
of the array the rest of the program need not change.

\prg{Array01A}
\prgl{array}{Array01A}

The program creates an array in the three steps described above.

\begin{itemize}
\item Initially, the variable \texttt{fib} of type integer array (denoted 
\texttt{int[]}) is allocated and contains the null value.
\item \texttt{new fib[7]} creates an array object with its seven fields having the default 
integer value zero; then the reference to the object is returned and 
stored in the variable \texttt{fib}.
\item The length field of the array is displayed above the cells for the elements.
\item A \texttt{for} loop is used to assign values to each element of the 
array.
\item The thin white lines show the constants and expressions that are used
as indices into the array.
\item \emph{Automatic dereferencing}: 
Although expressions like \texttt{fib[i-2]} seem to indicate that 
\texttt{fib} is being indexed, \texttt{fib} contains a reference to an array;
an implicit operation of dereferencing is carried out to obtain the
array itself from the reference and the index \texttt{[i-2]} is then
applied to that array.
\end{itemize}

\textbf{Concept} It is possible to combine the first two steps in creating an array: 
declaring the array field and allocating the array object.

\prg{Array01B}
\prgl{array}{Array01B}

This program combines the declaration of the
array field with its allocator. We have used the constant \texttt{SIZE} to 
specify the size of the array; this makes it easier to modify 
the program; nevertheless, \texttt{fib.length} is still used in the 
executable statements.
\begin{itemize}
  \item Initially, the static variable \texttt{SIZE} is created in the constant
  area and given its value.
  \item The execution of the program is as before.
\end{itemize}

\textbf{Exercise} Modify the program so that the fibonacci sequence appears in
reverse order.

\subsection{Computation within constructors}\label{con.02}

\textbf{Concept} Constructors are often used simply for assigning initial values to fields of an object; however, an arbitrary initializing computation can be carried out within the constructor.

\prg{Constructor02}\prgl{constructor}{Constructor02}

The price of a song will not change as long as the fields \texttt{second} and
\texttt{pricePerSecond} do not change; to avoid recomputing the price each time it is needed,
the class contains a field \texttt{price} whose value is computed \emph{within} the
constructor. The method \texttt{computePrice} is declared to be \texttt{private} 
because it is needed only by the constructor.
\begin{itemize}
\item The variable \texttt{song1} is allocated and contains the null value.
\item Memory is allocated for the fields of the object and default values are assigned to the three fields.
\item The constructor is called with three actual parameters; these values are assigned to the formal parameters of the constructor method and the values of the formal parameters are assigned to the three fields.
\item The method \texttt{computePrice} is called; it returns a value which stored in the field \texttt{price}.
\item The constructor returns a reference to the object, which is stored in the variable \texttt{song1}. The field \texttt{price} can be accessed to obtain the price of a song.
\end{itemize}

\textbf{Exercise}  Modify the class so that no song has a price greater than two 
currency units.

\subsection{Passing arrays as parameters}\label{array.03}

\textbf{Concept} An array is an object. Since the array variable itself 
contains a reference, it can be passed as an actual paramter to a method
and the reference is used to initialize the formal parameter.

\prg{Array03}\prgl{array}{Array03}

This program passes an array as a parameter to a method 
that reverses the elements of the array.
\begin{itemize}
\item Initially, the variable \texttt{fib} of type integer array is 
allocated. As part of the same statement, the array object is created with 
its seven fields having the values in the initializer; the reference to 
the object is returned and stored in the variable \texttt{fib}.
\item The array (that is, a \emph{reference} to the array) is passed as a parameter to the method \texttt{reverse}. There are now two arrows pointing to the array: the reference from the \texttt{main} method and the reference from the parameter \texttt{a} of the method \texttt{reverse}.
\item The method scans the first half of the array, exchanging each element with the corresponding one in the second half. Variables \texttt{i} and \texttt{j} contain the indices of the two elements that are exchanged.
\item Upon return from the method, the variable \texttt{fib} still 
contains a reference to the array, which has had its sequence of values 
reversed.
\end{itemize}

\textbf{Exercise} Instead of declaring the variable \texttt{j} outside the for-loop,
declare it just inside the for-loop as follows:
\begin{quote}
\texttt{int j = a.length-i-1;}
\end{quote}
Trace the execution and explain what happens.

\subsection{Returning an array from a method}\label{array.04}

\textbf{Concept} An array can be allocated within a method. Although the variable 
containing the reference to the array is local to the method, the array 
itself is global and the reference can be returned from the method.

\prg{Array04}\prgl{array}{Array04}

This program passes an array as a parameter to a method that reverses the elements of the array. The array is reversed into a new array \texttt{b} that is allocated in the method \texttt{reverse}. It is then returned to the main method and assigned to \texttt{reversedFib}, a  different variable of the same array type \texttt{int[]}.
\begin{itemize}
\item Initially, the variable \texttt{fib} of type integer array is 
allocated. As part of the same statement, the array object is created with 
its seven fields having the values in the initializer; the reference to 
the object is returned and stored in the variable \texttt{fib}.
\item A reference to the array is passed as a parameter to the method 
\texttt{reverse}. The formal parameter \texttt{a} contains a reference to the same array pointeed to by the actual parameter \texttt{fib}.
\item A new array \texttt{b} of the same type and length as the parameter \texttt{a} is declared and allocated.
\item Each iteration of the for-loop moves one element from the first half of \texttt{a} to the second half of \texttt{b} and one element from the second half of \texttt{a} to the first half of \texttt{b}. Variables \texttt{i} and \texttt{j} contain the indices of the two elements that are moved.
\item The reference to array \texttt{b} is returned.
Although array referenced by \texttt{b} was allocated \emph{within} the method call, it still exists after returning. 
\item The reference that is returned is assigned to \texttt{reversedFib}.
\end{itemize}

\textbf{Exercise} The program has a bug. Fix it!

\subsection{Explicit default constructors}\label{con.05}

\textbf{Concept} When no constructor is explicitly written in a class, a default implicit constructor with no parameters exists; this constructor does nothing. If, however, one or more explicit constructors are given, there is no longer a constructor with no parameters.
Should you want one, you have to write it explicitly.

\prg{Constructor05}
\prgl{constructor}{Constructor05}

This program includes an explicit constructor with no parameters that calls the constructor with three parameters to perform initialization.

\begin{itemize}
\item The variable \texttt{song1} is allocated and contains the null value.
\item Memory is allocated for the \emph{four} fields of the object and default values are assigned to the fields. 
\item The constructor is called with \emph{no} actual parameters; the call is resolved so that it is the second constructor that is executed. 
\item Three constant values are used to call the first constructor. The method name \texttt{this} means: call a constructor from \emph{this} class. 
This constructor initializes the first three fields from the parameters, and the value of the fourth field is computed by calling the method \texttt{computePrice}.
\item The constructor returns a reference to the object, which is stored in the variable \texttt{song1}.
\end{itemize}

\textbf{Exercise}  Modify the class so that the constructor without parameters 
obtains initial values from the input.

\subsection{Counting with for statements}\label{control.06}

\textbf{Concept} Although all loop structures can be programmed as \texttt{while} loops,
one special case is directly supported: writing a loop that executes a predetermined number
of times. The \texttt{for} statement has three parts:
\begin{quote}
\texttt{for (int i = 0; i < N; i++)}
\end{quote}
The first part declares a loop control variable and gives it an initial value.
The second part contains the exit condition: the loop body will be executed
as long as the expression evaluates to true. The third part describes how
the value of the control variable is modified after executing the loop body.
The syntax show is the conventional one for executing a loop \texttt{N} times.

\prg{Control06}
\prgl{control}{Control06}

This program computes the first six factorials in a \texttt{for} loop
and the last value is printed.

\begin{itemize}
\item The constant \texttt{N} and the variable \texttt{factorial} are allocated and initialized.
\item The control variable \texttt{i} is allocated and initialized.
\item The expression \texttt{i < N} is evaluated and evaluates to true.
\jel{} displays \texttt{Entering the for loop}.
\item The loop body is executed. 
\item The control variable is incremented as specified in the third part of the
\texttt{for} statement.
\item The previous three steps are repeated until the expression evaluates to false;
this causes the loop to be exited. \jel{} displays \texttt{Continuing the for loop}
as long as the expression evaluates to true, and \texttt{Exiting the for loop} when
it evaluates to false.
\item The final value of \texttt{factorial} is printed. 
\item \textbf{Important}: when the loop is exited, the control variable is
deallocated and no longer exists.
\end{itemize}

\textbf{Exercise} Rewrite the program using a \texttt{while} loop.


\subsection{Equals}\label{inher.07}

\textbf{Concept} There are two concepts of equality in Java: the \emph{operator} 
\texttt{==} compares primitives types and references, while the \emph{method}
\texttt{equals} compares objects. The default implementation of 
\texttt{equals} is like \texttt{==}, but it can be overridden in any class.

\prg{Inheritance07A} 
\prgl{inheritance}{Inheritance07A}

\begin{itemize}
\item Object \texttt{a1} of type \texttt{AParticle} is created.
\item \texttt{a1} is assigned to \texttt{a2} using \texttt{==}.
\item Object \texttt{a3} of type \texttt{AParticle} is created with the 
same values for its fields as the object referenced by \texttt{a1}.
\item Evaluating \texttt{a1==a2} returns \emph{true} because they both reference the same object.
\item Evaluating \texttt{a1==a3} returns \emph{false} because they reference different objects.
\item Strangely enough, evaluating \texttt{a1.equals(a3)} returns \emph{false}.
Although their fields are equal, the default implementation of \texttt{equals} is
the same as \texttt{==}!
\end{itemize}

\textbf{Exercise} Add the follow method to \texttt{AParticle} and run the
program again. What happens now?
\begin{verbatim}
   public boolean equals(AParticle a) {
        return this.position == a.position && this.spin == a.spin;
   }
\end{verbatim}

\bigskip

\prg{Inheritance07B}
\prgl{inheritance}{Inheritance07B}

Let us try to override the method \texttt{equals} in classes \texttt{BParticle}
and \texttt{CParticle}; the method returns true if the all fields of the
two objects are equal.
\begin{itemize}
\item Four objects are created: two equal objects \texttt{b1} and \texttt{b2}
of type \texttt{BParticle} and two unequal objects \texttt{c1} and \texttt{c2}
of type \texttt{CParticle}.
\item As expected, \texttt{b1.equals(b2)} returns \emph{true} and \texttt{c1.equals(c2)} returns \emph{false}.
\item \texttt{b1.equals(c1)} returns \emph{true}: since \texttt{CParticle}
is a subclass of \texttt{BParticle}, the variable \texttt{c1} is acceptable
as a parameter to the method \texttt{equals} declared in \texttt{BParticle}.
\texttt{c1} \emph{is} equal to \texttt{b1}, because we are only comparing 
the first two fields inherited from \texttt{BParticle} and these are equal.
\end{itemize}

\textbf{Exercise} Explain what happens if you try to evaluate \texttt{c1.equals(b1)}.

\bigskip

\prg{Inheritance07C}
\prgl{inheritance}{Inheritance07C}

It would be unusual for two objects to be considered equal if they
are of different types, even if one type is a subclass of another.
In fact, 
\begin{quote}
\texttt{public boolean equals(CParticle c)}
\end{quote}
does not override the method \texttt{equals} in \texttt{BParticle},
because an overriding method must have the \emph{same signature} as the overridden
method.

The method \texttt{equals} is declared in the root class \texttt{Object} as:
\begin{quote}
\texttt{public boolean equals(Object obj)}
\end{quote}
and this is the method that must be overridden.
This program shows the correct technique:
\begin{itemize}
	\item Since the parameter can now be any object, a check is first made that
	the parameter is not \texttt{null}.
	\item Similarly, a check is made that the parameter is of the same
	type as this object.
	\item Now that we know that the parameter is actually of this type,
	it can be cast from \texttt{Object} to the type.
	\item Only then is class-specific code performed---usually a field-by-field comparison.
\end{itemize}

Trace the execution of the program:
\begin{itemize}
\item Four objects are created: one object \texttt{b1} of type \texttt{BParticle} 
and three objects \texttt{c1}, \texttt{c2} and \texttt{c3} of type \texttt{CParticle}.
\item Clearly, comparing \texttt{c1} to \texttt{null} or \texttt{b1} returns \emph{false}.
\item Field-by-field comparisons are used if the parameter is of type \texttt{CParticle}: \texttt{c1.equals(c2)} returns \emph{false} and \texttt{c1.equals(c3)} returns \emph{true}.
\end{itemize}

\textbf{Exercise} Move the declaration of \texttt{equals} to class
\texttt{BParticle}, changing the code as needed. What now is the value of
\texttt{c1.equals(b1)}? Explain.

\subsection{Constructors with subclass object parameters}\label{con.08}

\textbf{Concept} An object of a subclass is also an object of the type of the 
superclass. Therefore, it can be used when an actual parameter is expected.

\prg{Constructor08}
\prgl{constructor}{Constructor08}

We allocate two objects, one of type \texttt{Song} and 
one of type \texttt{DiscountSong}, and use them as actual parameters in 
the constructor for an object of type \texttt{SongSet} that expects two 
parameters of type \texttt{Song}.
\begin{itemize}
\item Execute the program until the two objects one of type \texttt{Song} 
the other of type \texttt{DiscountSong} are 
allocated and their references assigned to the variables \texttt{song1} and \texttt{song2},
respectively.
(You may want to select \texttt{Animation / Run Until (ctrl-T)} to skip the 
animation of these declarations.)
\item The variable \texttt{set} is allocated, and an object of type 
\texttt{SongSet} is allocated with default null fields.
\item The constructor for \texttt{SongSet} is called and the references in the two variables
\texttt{song1} and \texttt{song2} are passed as actual parameters. These references 
are stored in the two fields \texttt{track1} and \texttt{track2}.
\item The reference to the object of class \texttt{SongSet} 
is returned and stored in \texttt{set}. 
\item The prices of the two objects are obtained and stored in the variables
\texttt{price1} and \texttt{price2}. \texttt{set} is an object 
of type \texttt{Songset}, while \texttt{set.track1} is an object of type 
\texttt{Song} and thus can be used to call the method 
\texttt{computePrice} of that class. Similarly for \texttt{price2},
except that \texttt{set.track2} is an object of type 
\texttt{DiscountSong}; check that the method \texttt{computePrice} of this
class is called.
\end{itemize}

\textbf{Exercise}  Can \texttt{s2} in the main method be declared to be of type
\texttt{Song}? Explain.


\subsection{Arrays of objects}\label{array.09}

\textbf{Concept} Arrays can contain references to arbitrary objects. There is no difference between these arrays and arrays whose values are of primitive type, except that an individual element can be of any type.

\prg{Array09} 
\prgl{array}{Array09}

Objects of class \texttt{Access} contain name of a bank customer and the access level permitted for that customer. This program creates two objects, assigns the their references to elements of the \texttt{Access} and then swaps the elements of the array.
\begin{itemize}
  \item The array \texttt{accessess} of type \texttt{Access[]} 
  is allocated and contains null references.
  \item An object of type \texttt{Access} are allocated and initialized by its constructor; a reference to the object is stored in the first element of the array \texttt{accessess}.
  \item Similarly, another object is created and stored in the second element.
  \item The array \texttt{accessess} is passed to the method \texttt{swap} along with the 
  indices 0 and 1.
  \item The two elements of the array are swapped. 
  Note that after executing \texttt{a[i] = a[j]}, 
  both elements of the array point to the second object (\texttt{Alice,4}), 
  while a reference to the first object (\texttt{Bob,3}) is saved in the variable \texttt{temp}.
\end{itemize}

\textbf{Exercise} Modify the program so that the initialization of the array
\texttt{accessess} is done in one statement instead of three.

\textbf{Exericse} Explain what happens if the method \texttt{swap} is replaced by: 

\hspace*{3em}\texttt{static void swap(Access a, Access b) \{}\\
\hspace*{6em}\texttt{Access temp = a;}\\
\hspace*{6em}\texttt{a = b;}\\
\hspace*{6em}\texttt{b = temp;}\\
\hspace*{3em}\texttt{\}}

and the call by:

\hspace*{3em}\texttt{swap(accesses[0], accesses[1]);}

