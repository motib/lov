\subsection{Continue statements}\label{control.08}

\textbf{Concept} The \texttt{break} statement is used to \emph{exit} a loop
from an arbitrary location in its body; the \texttt{continue} statement is used
to \emph{skip} the rest of a loop body and return to evaluate the condition
for continuing the loop.

\prg{Control08}
\prgl{control}{Control08}

This program sums all the positive integers less than \texttt{N} that are
divisible by $2$ or $3$ but not by both. For \texttt{N=10}, the result
is $2+3+4+8+9=26$.
\begin{itemize}
\item The constant \texttt{N} and the variable \texttt{sum} are allocated and initialized.
\item The \texttt{for} loop is standard and is executed for the values $0$ through $N-1$.
\item If \texttt{i} is divisible by $2$ and also by $3$ (for example, $6$), the 
\texttt{continue} statement is executed and the variable \texttt{sum} is not modified.
\item If \texttt{i} is divisible neither by $2$ nor by $3$ (for example, $5$), the 
\texttt{continue} statement is executed and the variable \texttt{sum} is not modified.
\item In all other cases, the value of \texttt{i} is added to \texttt{sum}.
\item The final value of \texttt{sum} is printed. 
\end{itemize}

\textbf{Exercise} Modify the program so that it explicitly checks for
divisibility by $6$, instead of checking for divisibility by $2$ and $3$ in
separate statements.

\textbf{Exercise} Modify the program so that \texttt{continue} is not used.

