\subsection{Conditional expressions}\label{control.02}

\textbf{Concept} A conditional expression is a shorthand for an if-statement that
assigns different values to one variable:
\begin{quote}
\texttt{if (expression) var = value1; else var = value2;}
\end{quote}
This can be rewritten more concisely as:
\begin{quote}
\texttt{var = (expression) ? value1 : value2;}
\end{quote}
The boolean-valued expression is evaluated:  
If the result is true, \texttt{value1} is assigned to \texttt{var}; if not, \texttt{value2} is assign to \texttt{var}.

\prg{Control02}
\prgl{control}{Control02}

The program computes the number of days in a month taking leap years into account.

\begin{itemize}
\item The variables are allocated and the first two, \texttt{year} and \texttt{month}, are
given initial values.
\item The expression \texttt{month == 2} evaluates to true, so the statement
following the expression is executed. \jel{} will display \texttt{Choosing then-branch} to 
emphasize this.
\item The inner statement is an assignment statement with a conditional expression.
The expression is evaluated
and its result is false, so the value after the colon is assigned to the variable.
\jel{} will display \texttt{Choosing else-branch}.
\item The value of \texttt{days} is printed.
\end{itemize}

\textbf{Exercise} Complete the program with the correct computation
for leap years: a year divisible by 100 is not a leap year unless it
is divisible by 400.

\textbf{Exercise} Rewrite the entire if-statement as nested conditional expressions.
