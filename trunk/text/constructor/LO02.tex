\subsection{Computation within constructors}\label{con.02}

\textbf{Concept} Constructors are often used simply for assigning initial values to fields of an object; however, an arbitrary initializing computation can be carried out within the constructor.

\prg{Constructor02}\prgl{constructor}{Constructor02}

The price of a song will not change as long as the fields \texttt{second} and
\texttt{pricePerSecond} do not change; to avoid recomputing the price each time it is needed,
the class contains a field \texttt{price} whose value is computed \emph{within} the
constructor. The method \texttt{computePrice} is declared to be \texttt{private} 
because it is needed only by the constructor.
\begin{itemize}
\item The variable \texttt{song1} is allocated and contains the null value.
\item Memory is allocated for the fields of the object and default values are assigned to the three fields.
\item The constructor is called with three actual parameters; these values are assigned to the formal parameters of the constructor method and the values of the formal parameters are assigned to the three fields.
\item The method \texttt{computePrice} is called; it returns a value which stored in the field \texttt{price}.
\item The constructor returns a reference to the object, which is stored in the variable \texttt{song1}. The field \texttt{price} can be accessed to obtain the price of a song.
\end{itemize}

\textbf{Exercise}  Modify the class so that no song has a price greater than two 
currency units.
