\subsection{Heterogeneous data structures}\label{inher.05}

\textbf{Concept} 
A heterogeneous data structure is one that can hold elements of different types. 
A data structure whose elements are of the type of a class can hold
references to objects of any subclass of that class.

\prg{Inheritance05}
\prgl{inheritance}{Inheritance05}

An array whose elements are of class \texttt{Particle} 
can store references to objects of any of its subclasses.
\begin{itemize}
  \item The objects are created and references to them assigned to elements
  of the elements of the array \texttt{p}.
  \item Method \texttt{newPosition} is invoked for the object referenced 
  by each element of the array \texttt{p}. 
  Check that the fields accessed are those of the object referenced by the
  array element and that the calls are dynamically dispatched to the method
  appropriate for the type of the object.
\end{itemize}

\textbf{Exercise} Every object in Java is a subclass of the class 
\texttt{Object}. Modify the program so that the variable \texttt{p} is of 
type array of \texttt{Object}.
