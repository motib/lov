\subsection{Equals}\label{inher.07}

\textbf{Concept} There are two concepts of equality in Java: the \emph{operator} 
\texttt{==} compares primitives types and references, while the \emph{method}
\texttt{equals} compares objects. The default implementation of 
\texttt{equals} is like \texttt{==}, but it can be overridden in any class.

\prg{Inheritance07A} 
\prgl{inheritance}{Inheritance07A}

\begin{itemize}
\item Object \texttt{a1} of type \texttt{AParticle} is created.
\item \texttt{a1} is assigned to \texttt{a2} using \texttt{==}.
\item Object \texttt{a3} of type \texttt{AParticle} is created with the 
same values for its fields as the object referenced by \texttt{a1}.
\item Evaluating \texttt{a1==a2} returns \emph{true} because they both reference the same object.
\item Evaluating \texttt{a1==a3} returns \emph{false} because they reference different objects.
\item Strangely enough, evaluating \texttt{a1.equals(a3)} returns \emph{false}.
Although their fields are equal, the default implementation of \texttt{equals} is
the same as \texttt{==}!
\end{itemize}

\textbf{Exercise} Add the follow method to \texttt{AParticle} and run the
program again. What happens now?
\begin{verbatim}
   public boolean equals(AParticle a) {
        return this.position == a.position && this.spin == a.spin;
   }
\end{verbatim}

\bigskip

\prg{Inheritance07B}
\prgl{inheritance}{Inheritance07B}

Let us try to override the method \texttt{equals} in classes \texttt{BParticle}
and \texttt{CParticle}; the method returns true if the all fields of the
two objects are equal.
\begin{itemize}
\item Four objects are created: two equal objects \texttt{b1} and \texttt{b2}
of type \texttt{BParticle} and two unequal objects \texttt{c1} and \texttt{c2}
of type \texttt{CParticle}.
\item As expected, \texttt{b1.equals(b2)} returns \emph{true} and \texttt{c1.equals(c2)} returns \emph{false}.
\item \texttt{b1.equals(c1)} returns \emph{true}: since \texttt{CParticle}
is a subclass of \texttt{BParticle}, the variable \texttt{c1} is acceptable
as a parameter to the method \texttt{equals} declared in \texttt{BParticle}.
\texttt{c1} \emph{is} equal to \texttt{b1}, because we are only comparing 
the first two fields inherited from \texttt{BParticle} and these are equal.
\end{itemize}

\textbf{Exercise} Explain what happens if you try to evaluate \texttt{c1.equals(b1)}.

\bigskip

\prg{Inheritance07C}
\prgl{inheritance}{Inheritance07C}

It would be unusual for two objects to be considered equal if they
are of different types, even if one type is a subclass of another.
In fact, 
\begin{quote}
\texttt{public boolean equals(CParticle c)}
\end{quote}
does not override the method \texttt{equals} in \texttt{BParticle},
because an overriding method must have the \emph{same signature} as the overridden
method.

The method \texttt{equals} is declared in the root class \texttt{Object} as:
\begin{quote}
\texttt{public boolean equals(Object obj)}
\end{quote}
and this is the method that must be overridden.
This program shows the correct technique:
\begin{itemize}
	\item Since the parameter can now be any object, a check is first made that
	the parameter is not \texttt{null}.
	\item Similarly, a check is made that the parameter is of the same
	type as this object.
	\item Now that we know that the parameter is actually of this type,
	it can be cast from \texttt{Object} to the type.
	\item Only then is class-specific code performed---usually a field-by-field comparison.
\end{itemize}

Trace the execution of the program:
\begin{itemize}
\item Four objects are created: one object \texttt{b1} of type \texttt{BParticle} 
and three objects \texttt{c1}, \texttt{c2} and \texttt{c3} of type \texttt{CParticle}.
\item Clearly, comparing \texttt{c1} to \texttt{null} or \texttt{b1} returns \emph{false}.
\item Field-by-field comparisons are used if the parameter is of type \texttt{CParticle}: \texttt{c1.equals(c2)} returns \emph{false} and \texttt{c1.equals(c3)} returns \emph{true}.
\end{itemize}

\textbf{Exercise} Move the declaration of \texttt{equals} to class
\texttt{BParticle}, changing the code as needed. What now is the value of
\texttt{c1.equals(b1)}? Explain.
