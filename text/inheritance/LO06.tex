\subsection{Abstract classes}\label{inher.06}

\textbf{Concept} Very often the ``root'' of a set of derived types has no 
meaning itself, in the sense that objects of that type would never be 
declared. For example, in a realistic simulation program, the would be no 
real particles that are just ``particles,'' only particles with names like 
$\alpha$-particles and $\beta$-particles.
An \emph{abstract} class can be declared which serves only as a root from 
which to derive a hierarchy of subclasses. 
It is not legal to declare \emph{objects} of an abstract class, 
although \emph{variables} of its type may be declared and used to 
reference objects of any type within the hierarchy. A method may also be 
declared abstract; this indicates that it \emph{must} be overridden in 
subclasses.

\prg{Inheritance06}
\prgl{inheritance}{Inheritance06}

The follow program declares \texttt{Particle} to
be \texttt{abstract} and no objects of that class can be declared. The
method \texttt{newPosition} is also declared abstrct because it doesn't
make sense to have a particle that you can't move.

\textbf{Exercise} The program does not compile successfully. Why? (Note that
in \jel{}, the problem is only found at when animating the program.)
Modify the program so that it compiles and executes.

\begin{itemize}
  \item The objects are created and references to them assigned to 
  elements of the array.
  \item Method \texttt{newPosition} is invoked for the object referenced 
  by each element of the array \texttt{p}. 
  Check that the fields accessed are those of the object referenced by the
  array element and that the calls are dynamically dispatched to the method
  appropriate for the type of the object.
\end{itemize}

\textbf{Exercise} It is possible to declare a nonabsract method in an 
abstract class. Give an example for this program, and explain why it is a 
reasonable thing to do.
