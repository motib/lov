\subsection{Constructors with object parameters}\label{con.07}

\textbf{Concept} An object can contain fields of other user-defined 
objects, not just of primitive and predefined types. There is no 
difference in the constructors, except that references to objects are
passed as actual parameters and assigned to fields of the object.

\prg{Constructor07}
\prgl{constructor}{Constructor07}

Two objects of type \texttt{Song} are allocated and 
assigned to fields of another object of type 
\texttt{SongSet} which has two fields of type \texttt{Song}.
\begin{itemize}
\item Execute the program until the two objects of type \texttt{Song} are 
allocated and their references assigned to the variables \texttt{song1} and \texttt{song2}.
(You may want to select \texttt{Animation / Run Until (ctrl-T)} to skip the 
animation of these declarations.)
\item A variable \texttt{set} is allocated. An object of type 
\texttt{SongSet} is allocated with default null fields.
\item The constructor for \texttt{SongSet} is called and the references in the two variables
\texttt{song1} and \texttt{song2} are passed as actual parameters. These references 
are stored in the two fields \texttt{track1} and \texttt{track2}.
\item The reference to the object of class \texttt{SongSet} 
is returned and stored in \texttt{set}. 
\item The prices of the two objects are obtained and stored in the variables
\texttt{price1} and \texttt{price2}. \texttt{set} is an object 
of type \texttt{Songset}, while \texttt{set.track1} is an object of type 
\texttt{Song} and thus can be used to call the method 
\texttt{computePrice}.
\end{itemize}

\textbf{Exercise}  Modify the program so that the variables \texttt{song1} and
\texttt{song2} are not used; instead, the constructors for the songs are
embedded within the constructor call for \texttt{SongSet}.

\textbf{Exercise}  Modify the program so the constructors for the songs are
call within the constructor for \texttt{SongSet}. Under what circumstances
would this be done?
