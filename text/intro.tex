\maketitle
\thispagestyle{empty}

%BEGIN LATEX
\vfill
%END LATEX

Copyright \copyright{} 2007 by Mordechai (Moti) Ben-Ari.

This work is licensed under the Creative Commons
Attribution-Noncommercial-ShareAlike 3.0 License. 
To view a copy of this license, 
visit \url{http://creativecommons.org/licenses/by-nc-sa/3.0/}; 
or, (b) send a letter to Creative Commons, 
543 Howard Street, 5th Floor, San Francisco, California, 94105, USA.

\clearpage

\section*{Introduction}
\emph{Learning objects (LOs)} are small, self-contained, reusable resources for 
learning. The advantages of LOs include: \emph{flexibility of 
use} (students can choose to work with LOs at their convenience) and 
\emph{adaptability} (students can choose to work only with those LOs that 
address topics they find difficult).

LOs are usually developed using multimedia authoring tools: see
the Codewitz project \url{http://codewitz.net/} and the LOs
developed by the Learning Technology Research Institute 
\url{http://www.londonmet.ac.uk/ltri/learningobjects/}.

LOV takes another approach for developing LOs for introductory programming:
the LOs are based on a visualization tool, here, the \jel{} system for animating
introductory programs in Java.

The LOs are primarily intended for self-study: to learn about a specific
concept in Java, work through the LO for that concept.

\section*{Learning objects for \jel{}}

The learning objects (LOs) described here consist of text and Java 
programs. Each LO is independent, so if you know the needed background 
material you can go directly to any LO. The learning objects are designed
for use with the \jel{} system for animating programs in Java:
\begin{center}
\url{http://cs.joensuu.fi/jeliot/}.
\end{center}

For each topic, a table is given that lists the LOs, the associated source 
files, and the ``prerequisites'' for each LO. The prerequisites are the number
of the LO that introduces \emph{concepts} that are assumed;
however, there is no need to actually work through the LOs in sequence.

The text for each LO starts with a description of the concept being presented 
and an overview of the example program. It is followed by a bulleted list for 
each program that describes what to observe as you \emph{step} through the 
program. The text for the LO ends with a programming exercise.

\vspace*{3ex}

\textbf{Acknowledgements:} I would like to thank Niko Myller and Andr\'es Moreno
for modifying \jel{} to accomodate the LOs, and Ronit Ben-Bassat Levy for suggestions
for improving the LOs.

\newpage

\section*{Tips for using \jel{}}

\begin{itemize}
\item The LOs have been tested with Jeliot Version 3.7.1; please ensure that you
are not using earlier versions.
\item Copy the source file directories to a clean directory so that if you make 
changes you will not modify the original files. 
Run \jel{} and open the source file for the LO you want to work with.
\item Learn how to use \jel{} before studying the LOs.
In particular, learn how to use \texttt{Step}, \texttt{Pause}, \texttt{Play},
and \texttt{Rewind} to control the animation.
\item Select \texttt{Animation / Run Until... (ctrl-T)} and enter a line number to begin
the animation at that line. This is very useful in two situations:
\begin{itemize}
	\item when you are animating a program several times and 
	wish to skip over the initialization or other parts of the code;
	\item when you wish to examine the final state after the last line of the
	main method: enter the line number of the closing brace of the main method.
\end{itemize}
\item Select \texttt{Options / Show History View} to enable storing of each step
of the animation; these can be viewed by selecting the \texttt{History} tab
on the right-hand side of the display. Enabling the history may slow Jeliot down,
especially for large programs.
\item The programs in the LOs use standard Java 
with two exceptions that simplify the animations:
\begin{itemize}
	\item None of the programs use the parameter of the \texttt{main} method.
	Since \jel{} accepts Java programs without the formal parameter 
	definition \texttt{String[] args}, the parameter has been commented-out in the programs.
	You can remove the comments to compile the programs with a Java compiler.
	
	If you wish to run \jel{} with the parameter, you can select \texttt{Options /
	Use Null Parameter to Call Main} to skip over the animation of the parameter.
	\item Two of the LOs on control structures use the input statement:
	\begin{center}
	\texttt{input = Input.nextInt()}
	\end{center}
	To compile these programs with standard Java, 
	add the following declaration to the \texttt{main} method:
    \begin{center}
	\texttt{java.util.Scanner Input = new java.util.Scanner(System.in);}
    \end{center}
\end{itemize}
\end{itemize}
