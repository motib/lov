\subsection{Counting with for statements}\label{control.06}

\textbf{Concept} Although all loop structures can be programmed as \texttt{while} loops,
one special case is directly supported: writing a loop that executes a predetermined number
of times. The \texttt{for} statement has three parts:
\begin{quote}
\texttt{for (int i = 0; i < N; i++)}
\end{quote}
The first part declares a loop control variable and gives it an initial value.
The second part contains the exit condition: the loop body will be executed
as long as the expression evaluates to true. The third part describes how
the value of the control variable is modified after executing the loop body.
The syntax show is the conventional one for executing a loop \texttt{N} times.

\prg{Control06}
\prgl{control}{Control06}

This program computes the first six factorials in a \texttt{for} loop
and the last value is printed.

\begin{itemize}
\item The constant \texttt{N} and the variable \texttt{factorial} are allocated and initialized.
\item The control variable \texttt{i} is allocated and initialized.
\item The expression \texttt{i < N} is evaluated and evaluates to true.
\jel{} displays \texttt{Entering the for loop}.
\item The loop body is executed. 
\item The control variable is incremented as specified in the third part of the
\texttt{for} statement.
\item The previous three steps are repeated until the expression evaluates to false;
this causes the loop to be exited. \jel{} displays \texttt{Continuing the for loop}
as long as the expression evaluates to true, and \texttt{Exiting the for loop} when
it evaluates to false.
\item The final value of \texttt{factorial} is printed. 
\item \textbf{Important}: when the loop is exited, the control variable is
deallocated and no longer exists.
\end{itemize}

\textbf{Exercise} Rewrite the program using a \texttt{while} loop.

