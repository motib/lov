\subsection{Switch statements}\label{control.09}

\textbf{Concept} A \texttt{switch} statement is a generalization of an \texttt{if} statement.
Instead of selecting between two alternatives depending on the value 
of a boolean-valued expression, an integer-valued expression is used and there can be
multiple alternatives introduced by the keyword \texttt{case}.
Since there are a very large number of 
integer values, an alternative labeled \texttt{default} is executed when the
value in the expression is not explicitly listed in one of the alternatives.

\textbf{Important:}
In an \texttt{if}-statement, the end of the statement (or block of statements)
of the first alternative causes a transfer of control 
to the end of the \texttt{if} statement, skipping over the statement (or block
of statements) in the second (\texttt{else}) alternative. This \emph{does not} happen in
a \texttt{switch}: control ``drops through'' from the end of one alternative to the beginning
of the next alternative.
A \texttt{break} statement must be used to transfer control from the end of an alternative
to the end of the \texttt{switch} statement.

\prg{Control09}
\prgl{control}{Control09}

This program computes the number of days in a month.
\begin{itemize}
\item The variables are allocated and the first two, \texttt{year} and \texttt{month}, are
given initial values.
\item The switch statement chooses a \texttt{case} depending on the value of the variable
\texttt{month}. \jel{} displays \texttt{Entering a switch statement}.
\item The \texttt{case} associated with \texttt{4} is selected.
\jel{} displays \texttt{This case is selected}.
The assignment statement assigns $30$ to \texttt{days}.
\item The assignment statement assigns $31$ to \texttt{days}.
\item The switch statement terminates and 
\jel{} displays \texttt{Exiting a switch statement}.
\item The value of \texttt{days} is printed.
\end{itemize}

\textbf{Exercise} Explain why the second assignment statement
is executed; fix the program.

\textbf{Exercise} Explain why the sequence of \texttt{case}'s for $4,6,9,11$ works.

\textbf{Exercise} Modify the program so that the \texttt{case}'s for the $31$-day months
are given explicitly and so that the days are computed correctly in leap years.
