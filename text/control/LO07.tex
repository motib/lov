\subsection{General for statements}\label{control.07}

\textbf{Concept} Arbitrary expressions can be given for the initial value of the \texttt{for} statement, the exit condition, and the modification of the control variable.

\prg{Control07}
\prgl{control}{Control07}

This program computes the sum of multiples of three that are less
than the square root of \texttt{N}.

\begin{itemize}
\item The constant \texttt{N} and the variable \texttt{sum} are allocated and initialized.
\item The control variable \texttt{i} is allocated and initialized.
\item The expression \texttt{i < Math.sqrt(N)} is evaluated and evaluates to true.
\jel{} displays \texttt{Entering the for loop}.
\item The loop body is executed. 
\item The control variable is incremented by three as specified in the third part of the
\texttt{for} statement.
\item The previous three steps are repeated until the expression evaluates to false;
this causes the loop to be exited. \jel{} displays \texttt{Continuing the for loop}
as long as the expression evaluates to true, and \texttt{Exiting the for loop} when
it evaluates to false.
\item The final value of \texttt{sum} is printed. 
\item \textbf{Important}: when the loop is exited, the control variable is
deallocated and no longer exists.
\end{itemize}

\textbf{Exercise} Is \texttt{for (;;;)} legal? If so, what does it mean?

\textbf{Exercise} Modify the program so that the square root is computed only once.
