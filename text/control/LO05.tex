\subsection{Break statements}\label{control.05}

\textbf{Concept} The exit from a while loop occurs \emph{before} the loop body and
the exit from a do-while loop occurs \emph{after} the loop body. 
The \texttt{break} statement can
be used to exit from an arbitrary location or locations from within the loop body.

The \texttt{break} statement is useful when the expression that leads to exiting the loop cannot
be evaluated until some statements from the loop body have been executed,
and yet there remain statements to be executed after the expression is evaluated.

\prg{Control05}
\prgl{control}{Control05}

The program sums a sequence of nonnegative integers read from the input and terminates when
a negative value is read.

\begin{itemize}
\item The two variables are allocated and \texttt{sum} is initialized with the value zero.
\item The \texttt{while} statement is executed with \texttt{true} as the loop expression.
Of course, \texttt{true} will never evaluate to false, so the loop will never be exited
at the \texttt{while}.
\item An integer value is read from the input. If it is negative the \texttt{break} statement
is executed and \jel{} displays \texttt{Exiting the while loop because of the break}.
\item Otherwise, the following assignment statement is executed and
\jel{} displays \texttt{Continuing without branching}.
\item After the assignment statement is executed, the loop starts again.
\end{itemize}

\textbf{Exercise} Write equivalent programs using a while loop
and a do-while loop.
