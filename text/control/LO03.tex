\subsection{While loops}\label{control.03}

\textbf{Concept} A loop enables the execution of a statement (including a block
of statements within braces) an arbitrary number of times.
This statement is called the \emph{loop body}.
In a while loop, an expression is evaluated \emph{before} each
execution of the loop body,
and loop body is executed if and only if the expression evaluates to true.

\prg{Control03}
\prgl{control}{Control03}

This program prints all factorials less than \texttt{LIMIT = 100}, namely,
$1!=1$, $2!=2$, $3!=6$, $4!=24$.
\begin{itemize}
\item The static constant and the two variables are allocated and initialized.
\item Then, and each time the keyword \texttt{while} is reached, the expression
is evaluated. If it is true, execution proceeds with the loop body, and
\jel{} displays \texttt{Entering the while loop} the first time
and \texttt{Continuing the while loop} on subsequent occasions.
\item The statements of the loop body are executed. 
They print the value of the current factorial, increment the counter
and compute the new factorial; then, control returns to the while-expression.
\item If and when the expression evaluates to false, execution
proceeds with the statement following the loop body.
\jel{} displays \texttt{Exiting the while loop}.
\end{itemize}

\textbf{Exercise} According to a formula by Euler,
\begin{displaymath}
\frac{1}{1^{2}} + \frac{1}{2^{2}} + \frac{1}{3^{2}} + \frac{1}{4^{2}} + \cdots 
= \frac{\pi^{2}}{6}    
\end{displaymath}
Write a program to compute the series until the difference between the two terms
is less than 0.1.
