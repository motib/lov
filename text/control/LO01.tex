\subsection{If-statements}\label{control.01}

\textbf{Concept} The execution of an if-statement starts with the evaluation of 
its boolean-valued expression. If the result is true, the statement written
after the closing parenthesis of the expression is executed; if the result is
false, the statement written after the \texttt{else} is executed. These statements
can be single statements or blocks of statements. In particular,
the statements can themselves be if-statements (\emph{nested if-statements}),
in which case the inner statement is executed the same way.

\prg{Control01}
\prgl{control}{Control01}

The program computes the number of days in a month taking leap years into account.

\begin{itemize}
\item The variables are allocated and the first two, \texttt{year} and \texttt{month}, are
given initial values.
\item The expression \texttt{month == 2} evaluates to false, so the statement
following the else is executed. \jel{} will display \texttt{Choosing else-branch} to 
emphasize this.
\item The inner statement is itself an if-statement. The expression is evaluated
and its result is true. Note that once one of the terms of \texttt{||} (or) becomes 
true, there is no need to evaluate the others.
\item The assignment statement following the statement is executed.
\jel{} will display \texttt{Choosing then-branch}. (The 
terminology \emph{then-branch} orginates from programming
languages that require the use of the keyword \texttt{then} between the expression
and the statement.)
\item The value of \texttt{days} is printed.
\end{itemize}

\textbf{Exercise} Complete the program with the correct computation
for leap years: a year divisible by 100 is not a leap year unless it
is also divisible by 400.

