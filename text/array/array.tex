\section{Learning Objects for Arrays}\label{s.arrays}

\textbf{Concept} An array is a sequence of elements of the same type;
the type of the elements can be a primitive types such as \texttt{int},
or a predefined or user-defined class type. To access an element of an array, an index is given; this may be any expression of type 
\texttt{int}, including an integer literal or a variable.

\begin{center}
\begin{tabular}{|c|l|l|c|}
\hline
LO & Topic  & Java Files (.java) & Prerequisites \\\hline
\ref{array.01} &  Array objects                       & Array01A, B &\\\hline
\ref{array.02} &  Array initializers                  & Array02  &  1\\\hline
\ref{array.03} &  Passing arrays as parameters        & Array03  &  2\\\hline 
\ref{array.04} &  Returning an array from a method    & Array04  &  2\\\hline
\ref{array.05} &  Array assignment can create garbage & Array05  &  4\\\hline
\ref{array.06} &  Two-dimensional arrays              & Array06  &  3\\\hline
\ref{array.07} &  Arrays of arrays                    & Array07  &  6\\\hline
\ref{array.08} &  Ragged arrays                       & Array08  &  6\\\hline
\ref{array.09} &  Arrays of objects                   & Array09  &  3\\\hline
\end{tabular}
\end{center}

The example used in LO~\ref{array.01} through LO~\ref{array.01} is to fill an array 
with a sequence of fibonacci numbers (0,1,1,2,3,5,8). The programs for 
LO~\ref{array.05} through LO~\ref{array.08} concern matrices. 
The program for LO~\ref{array.09} is explained there.

\subsection{A void method}\label{method.01}

\textbf{Concept} When a method that is declared \texttt{void} is called,
it allocates memory for its parameters and local variables, executes
its statements and then returns. The call is a statement constructed
from the name of the method followed by a list of actual parameters.

\prg{Method01}
\prgl{method}{Method01}

The program computes the maximum of two integer values.

\begin{itemize}
\item The variables \texttt{x} and \texttt{y} are allocated and initialized.
\item The method is called with the values of the actual parameters \texttt{x} and \texttt{y}.
\item Memory is allocated for the formal parameters of the method and the local variables.
This is called an \emph{activation record} and is displayed by \jel{} in the upper left
hand part of the screen labeled \texttt{Method Area}. The new activation record hides
the previous ones which are no longer accessible.
\item The actual parameters are used to initialize the formal parameters in the activation
record.
\item The local variable \texttt{max} is allocated within the activation record.
\item The statements of the method are executed.
\item After the last statement has been executed, the method \emph{returns}
and the activation record is deallocated.
\item Execution continues with the statement after the method call.
Here, the method is called again, this time with an integer literal
as an actual parameter instead of a variable. 
\end{itemize}
\textbf{Note:} In a call to a static method, the name of the 
class in which it is defined can be given as in the second call.
Since the method is defined in the \emph{same} class as the call,
the class name need not be given, as shown in the first call.

\textbf{Exercise} Trace the execution of a call of the following method and
explain why it doesn't swap the values of the actual parameters.
\begin{quote}
\texttt{void swap(int a, int b) \{}\\
\hspace*{2em}\texttt{int temp = a; a = b; b = temp;}\\
\texttt{\}}
\end{quote}
Can you write a method to swap two integer values?
\subsection{A method returning a value}\label{method.02}

\textbf{Concept} When a method that is declared with a return type is called,
it allocates memory for its parameters and local variables, executes
its statements and then returns a value of the type. 
The call is a statement constructed
from the name of the method followed by a list of actual parameters;
the call is an expression and can appear wherever an expression is allowed.

\prg{Method02}
\prgl{method}{Method02}

\begin{itemize}
\item The variables \texttt{x} and \texttt{y} are allocated and initialized;
the variable \texttt{max} is allocated but not initialized.
\item An assignment statement is executed: the expression on the right hand side is
a method call including the values of the actual parameters \texttt{x} and \texttt{y}.
\item Memory is allocated for the formal parameters of the method and the local variables.
This is called an \emph{activation record} and is displayed by \jel{} in the upper left
hand part of the screen labeled \texttt{Method Area}. The new activation record hides
the previous ones which are no longer accessible.
\item The actual parameters are used to initialize the formal parameters in the activation
record.
\item The statements of the method are executed.
\item When the statement \texttt{return b} is executed, the value of \texttt{b} is used
for the value to be returned.
\item The method \emph{returns} and the activation record is deallocated.
\item The value returned becomes the value of the expression assigned to the
variable \texttt{max}.
\item The value of \texttt{max} is printed.
\end{itemize}

\textbf{Exercise} Write the body of the main method as one statement.

\subsection{Calling one method from another}\label{method.03}

\textbf{Concept} One method can call another, that is, when executing
one method, any statement or expression call be a method call. A sequence
of method calls results in a \emph{stack} of activation records,
where each method (except the last one that was called) 
is waiting for the method it called to return. 
There is no limit on the \emph{depth} of method calls,
except of course the amount of memory allocated to the program.

\textbf{Note:} The \texttt{main} method is a method like any other. 
The operating system can be considered as a program which calls the main method.
This call has a single parameter: an array of strings containing the contents
of the command line.

\prg{Method03}
\prgl{method}{Method03}

\begin{itemize}
\item The \texttt{main} method calls the method \texttt{printMax};
the actual parameters are two integer literals.
\item The activation record for \texttt{printMax} is allocated,
and the actual parameters are used to initialize the formal parameters 
\texttt{a} and \texttt{b}.
\item The variable \texttt{max} is allocated but not initialized.
\item The method \texttt{maximum} is called; the actual parameters are the
values of \texttt{a} and \texttt{b}, 
which are the formal variables of method \texttt{printMax}.
\item An activation record is allocated for \texttt{maximum}. (There are now
three activation in the stack.) The new activation record includes
memory for the formal parameters \texttt{a} and \texttt{b}; note that these are
new parameters not at all related to the formal parameters of the same names
in the previous method \texttt{printMax} because those parameters are hidden.
\item The method \texttt{maximum} executes its body and returns a value.
Just before it returns, select the tab \texttt{Call Tree} above the graphic display;
the sequence of calls from \texttt{main} 
to \texttt{printMax} and then \texttt{maximum} is displayed.
Select \texttt{Theater} to return to the animated display.
\item When the method returns, its activation record is deallocated, uncovering the
activation record of \texttt{printMax}. 
\item The value returned is assigned to the variable \texttt{max} and printed.
\item When \texttt{printMax} completes its execution, its activation record
is deallocated.
\end{itemize}
\textbf{Note:} In a call to a static method, the name of the 
class in which it is defined can be given as in the call to \texttt{maximum}.
Since the method is defined in the same class as the call,
the class name need not be given, as shown in the call to \texttt{printMax}.

\textbf{Exercise} Write a program to compute the maximum of six values
using as few statements as possible.

\subsection{Returning an array from a method}\label{array.04}

\textbf{Concept} An array can be allocated within a method. Although the variable 
containing the reference to the array is local to the method, the array 
itself is global and the reference can be returned from the method.

\prg{Array04}\prgl{array}{Array04}

This program passes an array as a parameter to a method that reverses the elements of the array. The array is reversed into a new array \texttt{b} that is allocated in the method \texttt{reverse}. It is then returned to the main method and assigned to \texttt{reversedFib}, a  different variable of the same array type \texttt{int[]}.
\begin{itemize}
\item Initially, the variable \texttt{fib} of type integer array is 
allocated. As part of the same statement, the array object is created with 
its seven fields having the values in the initializer; the reference to 
the object is returned and stored in the variable \texttt{fib}.
\item A reference to the array is passed as a parameter to the method 
\texttt{reverse}. The formal parameter \texttt{a} contains a reference to the same array pointeed to by the actual parameter \texttt{fib}.
\item A new array \texttt{b} of the same type and length as the parameter \texttt{a} is declared and allocated.
\item Each iteration of the for-loop moves one element from the first half of \texttt{a} to the second half of \texttt{b} and one element from the second half of \texttt{a} to the first half of \texttt{b}. Variables \texttt{i} and \texttt{j} contain the indices of the two elements that are moved.
\item The reference to array \texttt{b} is returned.
Although array referenced by \texttt{b} was allocated \emph{within} the method call, it still exists after returning. 
\item The reference that is returned is assigned to \texttt{reversedFib}.
\end{itemize}

\textbf{Exercise} The program has a bug. Fix it!

\subsection{Array assignment can create garbage}\label{array.05}

\textbf{Concept} Since an array variable contains a reference to the array itself, if 
\texttt{null} or another value (another array of the same type) is assigned to the 
variable, the first array may no longer be accessible. Inaccessible memory 
is called \emph{garbage}. The Java runtime system includes a \emph{garbage 
collector} whose task is to return garbage to the pool of memory that can 
be allocated.

\prg{Array05}\prgl{array}{Array05}

An array referenced by the variable \texttt{fib} is passed as a parameter to a method that moves the values of the elements in the first half of the array \texttt{fib} into a new array \texttt{b} which is allocated in the method. The new array is returned to the main method and assigned to the variable \texttt{fib}, destroying the reference to the original array.

\begin{itemize}
\item Initially, the variable \texttt{fib} of type integer array is allocated. As part of the same statement, the array object is created with its seven fields having the values in the initializer; the reference to the object is returned and stored in the variable \texttt{fib}.
\item A reference to the array is passed as a parameter to the method \texttt{first}. The formal parameter \texttt{a} contains a reference to the same array pointed to by the actual parameter \texttt{fib}.
\item A new array \texttt{b} of the same type as the parameter \texttt{a} but half the length is declared and allocated.
\item Each iteration of the for-loop moves one element from the first half of \texttt{a} to the corresponding element in the array \texttt{b}.
\item The reference to array \texttt{b} is returned. Although array referenced by \texttt{b} was allocated \emph{within} the method call, it still exists after returning. 
\item There are no references to the original array so it is inaccessible. \jel{} does not visualize garbage collection so the array remains visualized in the Instance and Array Area until the end of the program.
\end{itemize}

\textbf{Exercise} Modify the program so that the original array remains accessible
in a different field.

\subsection{Constructors for subclasses}\label{con.06}

\textbf{Concept} Constructors are \emph{not} inherited. You must 
explicitly define a constructor for a subclass (with or without 
parameters). As its first statement, the constructor for the subclass must 
call a constructor for the superclass using the method \texttt{super}.

\prg{Constructor06A}
\prgl{constructor}{Constructor06A}

The website wants to sell certain songs at a discount.
The subclass \texttt{DiscountSong} inherits from class 
\texttt{Song}, adds a field \texttt{discount} and overrides 
\texttt{computePrice} to include \texttt{discount} in the computation. 
The constructor for the subclass calls the three-parameter constructor for 
the superclass, passing it the three parameters that it expects. The 
fourth parameter is used directly in the constructor \texttt{DiscountSong} 
to initialize the field \texttt{discount}.
\begin{itemize}
\item The variable \texttt{song1} is allocated and contains the null value.
\item Memory is allocated for the \emph{five} fields of the object of the subclass \texttt{DiscountSong} and default values are assigned to the fields. 
Four fields inherited from the superclass and one field \texttt{discount} 
added by the subclass.
\item The constructor for the subclass \texttt{DiscountSong} is called 
with four parameters. It calls the constructor for the superclass 
\texttt{Song} which assigns values to three fields from the parameters and 
the fourth by calling \texttt{computePrice}.
\item The superclass constructor returns and then the fourth parameter of 
the subclass constructor is assigned to the field \texttt{discount}.
\item The reference to the subclass object is returned and assigned to a 
variable \texttt{song1} of that type.
\end{itemize}
Unfortunately, this does not do what we intended, because the superclass 
method for \texttt{computePrice} is used to compute \texttt{price} 
instead of the method from the subclass.

\textbf{Exercise} Could \texttt{song1} be declared to be of type
\texttt{Song}? Explain your answer.

\bigskip

\prg{Constructor06B}
\prgl{constructor}{Constructor06B}

The problem can be solved by adding a call to 
\texttt{computePrice} in the constructor for the subclass. 

Check this by executing the code and ensuring that the discounted price is computed.

The disadvantage of this solution is that we are calling 
\texttt{computePrice} twice.

\bigskip

\prg{Constructor06C}
\prgl{constructor}{Constructor06C}

Normally in an object-oriented program, all the 
fields of an object are private and an accessor method like 
\texttt{getPrice()} is used to access the values of the fields.
If this is done, the computation of the price can be placed in 
the accessor for the superclass and overridden in accessor for the 
subclass. 

Check this by executing the code and ensuring that the discounted price is computed.

The disadvantage of this solution is that the computation is 
performed for each access of the field \texttt{price}.

\textbf{Exercise} Develop other solutions for this problem: (a) Call
\texttt{computePrice} explicitly after the call to the constructor; (b)
Modify \texttt{getPrice} to compute the value of \texttt{price} on the first
call and save it for future calls. Summarize the advantages and
disadvantages of all the solutions for this problem.

\subsection{Arrays of arrays}\label{array.07}

\textbf{Concept} A two-dimensional array is really an array of arrays; that is, each 
element of the array contains a reference to another array. Therefore, by
using only one index a one-dimensional array is obtained.

\prg{Array07}
\prgl{array}{Array07}

This program creates a $2 \times 2$ matrix and then assigns the second row to a variable of type \emph{one-dimensional} array.
\begin{itemize}
  \item A two-dimensional array is allocated, the reference to it is assigned to the variable \texttt{matrix}.
  The variable \texttt{matrix} contains one reference for each row and the rows are allocated as separate objects. Note that \jel{} displays rows from top to bottom as it does for all objects!
  \item The elements of the array are initialized to (0,1,2,3) in a nested for-loop.
  The outside loop iterates over the rows and the inner loop iterates over the
  columns within an array.
  \item \texttt{matrix.length} is used to get the number of rows and 
  \texttt{matrix[i].length} to get the number of columns in row \texttt{i}, which 
  is the same for all rows in this program.
  \item A variable \texttt{vector} of type one-dimensional array is declared and initialized
  with the second row of the matrix, \texttt{matrix[1]}.
  \end{itemize}

\textbf{Exercise} Write a program to rotate the rows of the array \texttt{matrix}. That is, row 0 becomes row 1 and row 1 becomes row 0. Now do this for an array of size $3 \times 3$: row 0 becomes row 1, row 1 becomes row 2 and row 2 becomes row 0.

\subsection{Constructors with subclass object parameters}\label{con.08}

\textbf{Concept} An object of a subclass is also an object of the type of the 
superclass. Therefore, it can be used when an actual parameter is expected.

\prg{Constructor08}
\prgl{constructor}{Constructor08}

We allocate two objects, one of type \texttt{Song} and 
one of type \texttt{DiscountSong}, and use them as actual parameters in 
the constructor for an object of type \texttt{SongSet} that expects two 
parameters of type \texttt{Song}.
\begin{itemize}
\item Execute the program until the two objects one of type \texttt{Song} 
the other of type \texttt{DiscountSong} are 
allocated and their references assigned to the variables \texttt{song1} and \texttt{song2},
respectively.
(You may want to select \texttt{Animation / Run Until (ctrl-T)} to skip the 
animation of these declarations.)
\item The variable \texttt{set} is allocated, and an object of type 
\texttt{SongSet} is allocated with default null fields.
\item The constructor for \texttt{SongSet} is called and the references in the two variables
\texttt{song1} and \texttt{song2} are passed as actual parameters. These references 
are stored in the two fields \texttt{track1} and \texttt{track2}.
\item The reference to the object of class \texttt{SongSet} 
is returned and stored in \texttt{set}. 
\item The prices of the two objects are obtained and stored in the variables
\texttt{price1} and \texttt{price2}. \texttt{set} is an object 
of type \texttt{Songset}, while \texttt{set.track1} is an object of type 
\texttt{Song} and thus can be used to call the method 
\texttt{computePrice} of that class. Similarly for \texttt{price2},
except that \texttt{set.track2} is an object of type 
\texttt{DiscountSong}; check that the method \texttt{computePrice} of this
class is called.
\end{itemize}

\textbf{Exercise}  Can \texttt{s2} in the main method be declared to be of type
\texttt{Song}? Explain.


\subsection{Returning locally instantiated objects}\label{method.09}

\textbf{Concept} When a method terminates, its activation record is deallocated.
However, if an object has been instantiated \emph{within the method},
a reference to the object can be returned to the calling method.

\prg{Method09}
\prgl{method}{Method09}

This program computes the cost of a song as the product of its length
in seconds and the price per second. A class \texttt{Song} is defined
to encapsulate the field \texttt{seconds} and the method \texttt{computePrice}.
The method \texttt{double} in the \texttt{main} method receives a reference
to an object of class \texttt{Song} as a parameter and returns a reference
to an new object of class \texttt{Song} whose field \texttt{seconds} is twice as large.

\begin{itemize}
\item Two objects of class Song are instantiated and references to them
are assigned to the variables \texttt{song1} and \texttt{song2}.
\item The method \texttt{doubleSong} is called with an actual parameter
that is a reference \texttt{song1} to an object of class \texttt{Song}.
The actual parameter is used to initialize the formal
parameter; check that \texttt{song1} and \texttt{s1} reference the same object.
\item \emph{Within the method}, the \texttt{seconds} field of the object referenced
by \texttt{s1} is used to instantiate a new object whose reference is assigned
to the variable \texttt{d} of class \texttt{Song}.
\item The method returns the reference to the object contained in \texttt{d};
although \texttt{d} disappears when the activation record is deallocated,
the object still exists as does the reference that is returned.
\item The returned reference is assigned to the variable \texttt{longSong};
check that \texttt{song1} and \texttt{longSong} reference \emph{different} objects!
\end{itemize}

\textbf{Exercise} Replace the last line of the program by:
\begin{quote}
\texttt{song1 = doubleSong(song1);}
\end{quote}
and explain precisely what happens.


