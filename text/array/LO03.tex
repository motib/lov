\subsection{Passing arrays as parameters}\label{array.03}

\textbf{Concept} An array is an object. Since the array variable itself 
contains a reference, it can be passed as an actual paramter to a method
and the reference is used to initialize the formal parameter.

\prg{Array03}\prgl{array}{Array03}

This program passes an array as a parameter to a method 
that reverses the elements of the array.
\begin{itemize}
\item Initially, the variable \texttt{fib} of type integer array is 
allocated. As part of the same statement, the array object is created with 
its seven fields having the values in the initializer; the reference to 
the object is returned and stored in the variable \texttt{fib}.
\item The array (that is, a \emph{reference} to the array) is passed as a parameter to the method \texttt{reverse}. There are now two arrows pointing to the array: the reference from the \texttt{main} method and the reference from the parameter \texttt{a} of the method \texttt{reverse}.
\item The method scans the first half of the array, exchanging each element with the corresponding one in the second half. Variables \texttt{i} and \texttt{j} contain the indices of the two elements that are exchanged.
\item Upon return from the method, the variable \texttt{fib} still 
contains a reference to the array, which has had its sequence of values 
reversed.
\end{itemize}

\textbf{Exercise} Instead of declaring the variable \texttt{j} outside the for-loop,
declare it just inside the for-loop as follows:
\begin{quote}
\texttt{int j = a.length-i-1;}
\end{quote}
Trace the execution and explain what happens.
