\subsection{Calling one method from another}\label{method.03}

\textbf{Concept} One method can call another, that is, when executing
one method, any statement or expression call be a method call. A sequence
of method calls results in a \emph{stack} of activation records,
where each method (except the last one that was called) 
is waiting for the method it called to return. 
There is no limit on the \emph{depth} of method calls,
except of course the amount of memory allocated to the program.

\textbf{Note:} The \texttt{main} method is a method like any other. 
The operating system can be considered as a program which calls the main method.
This call has a single parameter: an array of strings containing the contents
of the command line.

\prg{Method03}
\prgl{method}{Method03}

\begin{itemize}
\item The \texttt{main} method calls the method \texttt{printMax};
the actual parameters are two integer literals.
\item The activation record for \texttt{printMax} is allocated,
and the actual parameters are used to initialize the formal parameters 
\texttt{a} and \texttt{b}.
\item The variable \texttt{max} is allocated but not initialized.
\item The method \texttt{maximum} is called; the actual parameters are the
values of \texttt{a} and \texttt{b}, 
which are the formal variables of method \texttt{printMax}.
\item An activation record is allocated for \texttt{maximum}. (There are now
three activation in the stack.) The new activation record includes
memory for the formal parameters \texttt{a} and \texttt{b}; note that these are
new parameters not at all related to the formal parameters of the same names
in the previous method \texttt{printMax} because those parameters are hidden.
\item The method \texttt{maximum} executes its body and returns a value.
Just before it returns, select the tab \texttt{Call Tree} above the graphic display;
the sequence of calls from \texttt{main} 
to \texttt{printMax} and then \texttt{maximum} is displayed.
Select \texttt{Theater} to return to the animated display.
\item When the method returns, its activation record is deallocated, uncovering the
activation record of \texttt{printMax}. 
\item The value returned is assigned to the variable \texttt{max} and printed.
\item When \texttt{printMax} completes its execution, its activation record
is deallocated.
\end{itemize}
\textbf{Note:} In a call to a static method, the name of the 
class in which it is defined can be given as in the call to \texttt{maximum}.
Since the method is defined in the same class as the call,
the class name need not be given, as shown in the call to \texttt{printMax}.

\textbf{Exercise} Write a program to compute the maximum of six values
using as few statements as possible.
