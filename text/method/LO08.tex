\subsection{Returning objects}\label{method.08}

\textbf{Concept} A return value can be a reference to an object.

\prg{Method08}
\prgl{method}{Method08}

This program computes the cost of a song as the product of its length
in seconds and the price per second. A class \texttt{Song} is defined
to encapsulate the field \texttt{seconds} and the method \texttt{computePrice}.
The method \texttt{longer} in the \texttt{main} method receives references
to two objects of class \texttt{Song} as parameters and returns a reference
to the one with the larger value of the field \texttt{seconds}.

\begin{itemize}
\item Two objects of class Song are instantiated and references to them
are assigned to the variables \texttt{song1} and \texttt{song2}.
\item The method \texttt{longer} is called with two parameters
that are references to objects of class \texttt{Song}.
The actual parameters are used to initialize the formal
parameters; check that \texttt{song1} and \texttt{s1} reference the same object,
as do \texttt{song2} and \texttt{s2}.
\item Since the formal parameters \texttt{s1} and \texttt{s2} receive references to objects
of class \texttt{Song}, they can be used to access the fields \texttt{seconds} of each object.
\item The method returns the reference to the object whose field \texttt{seconds}
has the larger value. The reference is assigned to the variable \texttt{longerSong};
check that this reference is to the same object as the reference in \texttt{song1}.
\item The reference in \texttt{longerSong} is used to call the method \texttt{computePrice}
and the value returned is assigned to the variable \texttt{price2}.
\item The value \texttt{price2} is printed.
\end{itemize}

\textbf{Exercise} Modify the program so that discount does not use the explicit
parameter \texttt{s}.

\textbf{Exercise} Replace the last declaration and statements of the program by one
declaration.

\textbf{Exercise} Write a method to swap two integer values.
